\documentclass[12pt]{article}
\usepackage[utf8]{inputenc}
\usepackage[T1]{fontenc}
\usepackage{amsmath,amsfonts,amssymb}
\usepackage{graphicx}
\usepackage{a4wide}
\usepackage{hyperref}

\begin{document}

\paragraph{Title:} Deep Learning Models for Acne Recognition

\paragraph{Abstract:} Acne is a common skin problem that affects many people. To diagnose it, it is important to assess the severity of the condition. Dermatologists often use different scales and criteria to do this. It is proposed to use various deep learning models to speed up this process. The main task is to develop and evaluate detection models for acne recognition.

\paragraph{Datasets:}  A brief description of data in the computational experiment and. Links to the datasets. The datasets shall be open-source. The data shall be ready-to-model.
\begin{enumerate}
\item ACNE04 dataset with labels and bounding boxes that was proposed in ~\cite{wu2019joint}.
\item The dataset of self-portraits of individuals with labels.
\end{enumerate}

\paragraph{References:}  Papers with a fast intro and the basic solution to compare.
\begin{enumerate}
\item The formulation of the problem in the medical domain: ~\cite{https://doi.org/10.1046/j.1365-4362.1997.00099.x}.
\item A baseline: \url{https://arxiv.org/abs/1907.07901}.
\end{enumerate}

\paragraph{Basic solution:} The code of the baseline algorithm is proposed in \url{https://github.com/microsoft/nestle-acne-assessment}.

\paragraph{Authors:} Angelina Holicheva

\bibliographystyle{unsrt}
\bibliography{Holicheva-Step-9}
\end{document}